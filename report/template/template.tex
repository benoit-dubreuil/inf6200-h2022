\documentclass{article}

\usepackage{arxiv}

\usepackage[french]{babel}  % Décommenter pour écrire en français
\usepackage[utf8]{inputenc} % allow utf-8 input
\usepackage[T1]{fontenc}    % use 8-bit T1 fonts
\usepackage{hyperref}       % hyperlinks
\usepackage{url}            % simple URL typesetting
\usepackage{booktabs}       % professional-quality tables
\usepackage{amsfonts}       % blackboard math symbols
\usepackage{nicefrac}       % compact symbols for 1/2, etc.
\usepackage{microtype}      % microtypography
\usepackage{lipsum}

\title{
  Rapport final \\
  INF6200, Initiation à la recherche}


\author{
  Benoît Dubreuil \\
  Département d'informatique \\ % TODO : Garder l'affiliation universitaire, même si je suis un étudiant au bac?
  Université du Québec à Montréal \\
  Montréal, Canada H3C 3P8 \\
  \texttt{dubreuil.benoit.2@courrier.uqam.ca} \\
  \And
  Joël Lefebvre \\ % TODO : Garder superviseur de recherche dans la section des auteurs?
  Département d'informatique\\
  Université du Québec à Montréal \\
  Montréal, Canada H3C 3P8\\
  \texttt{lefebvre.joel@uqam.ca} \\
}

\begin{document}
  \maketitle

  \begin{abstract}
    À compléter : un résumé de l'article en 300-400 mots. / To complete: the paper abstract in 300-400 words.
  \end{abstract}


  \section{Introduction} % EN / FR
  À compléter / To complete


  \section{Methods} % EN
%\section{Méthodologie} % FR
  À compléter / To complete


  \section{Results} % EN
%\section{Résultats} % FR
  À compléter / To complete


  \section{Discussion} % EN/FR
  À compléter / To complete


  \section{Conclusion} % EN/FR
  À compléter / To complete

  \section*{Acknowledgments} % EN
%\section*{Remerciements} % FR
  À compléter / To complete

% Ajoutez vos références (format bibtex) dans le fichier references.bib
  \bibliographystyle{unsrt}
  \bibliography{references}

%%%%%%%%%%%%%%%%%%%%%%%%%%%%%%

% EN: The following sections can be removed / commented. They were kept to serve as LaTeX examples.
% FR: Les prochaines sections peuvent être retirées ou commentées. Elles ont été conservées pour servir d'exemples LaTeX.


  \section{Headings: first level}
  \label{sec:headings}

  \lipsum[4] See Section \ref{sec:headings}.

  \subsection{Headings: second level}
  \lipsum[5]
  \begin{equation}
    \xi _{ij}(t)=P(x_{t}=i,x_{t+1}=j|y,v,w;\theta)= {\frac {\alpha _{i}(t)a^{w_t}_{ij}\beta _{j}(t+1)b^{v_{t+1}}_{j}(y_{t+1})}{\sum _{i=1}^{N} \sum _{j=1}^{N} \alpha _{i}(t)a^{w_t}_{ij}\beta _{j}(t+1)b^{v_{t+1}}_{j}(y_{t+1})}}
  \end{equation}

  \subsubsection{Headings: third level}
  \lipsum[6]

  \paragraph{Paragraph}
  \lipsum[7]


  \section{Examples of citations, figures, tables, references}
  \label{sec:others}
  \lipsum[8] \cite{kour2014real,kour2014fast} and see \cite{hadash2018estimate}.

  The documentation for \verb+natbib+ may be found at
  \begin{center}
    \url{http://mirrors.ctan.org/macros/latex/contrib/natbib/natnotes.pdf}
  \end{center}
  Of note is the command \verb+\citet+, which produces citations
  appropriate for use in inline text. For example,
  \begin{verbatim}
   \citet{hasselmo} investigated\dots
  \end{verbatim}
  produces
  \begin{quote}
    Hasselmo, et al.\ (1995) investigated\dots
  \end{quote}

  \begin{center}
    \url{https://www.ctan.org/pkg/booktabs}
  \end{center}

  \subsection{Figures}
  \lipsum[10]
  See Figure \ref{fig:fig1}. Here is how you add footnotes. \footnote{Sample of the first footnote.}
  \lipsum[11]

  \begin{figure}
    \centering
    \fbox{\rule[-.5cm]{4cm}{4cm} \rule[-.5cm]{4cm}{0cm}}
    \caption{Sample figure caption.}
    \label{fig:fig1}
  \end{figure}

  \subsection{Tables}
  \lipsum[12]
  See awesome Table~\ref{tab:table}.

  \begin{table}
    \caption{Sample table title}
    \centering
    \begin{tabular}{lll}
      \toprule
      \multicolumn{2}{c}{Part} \\
      \cmidrule(r){1-2}
      Name     & Description     & Size ($\mu$m) \\
      \midrule
      Dendrite & Input terminal  & $\sim$100     \\
      Axon     & Output terminal & $\sim$10      \\
      Soma     & Cell body       & up to $10^6$  \\
      \bottomrule
    \end{tabular}
    \label{tab:table}
  \end{table}

  \subsection{Lists}
  \begin{itemize}
    \item Lorem ipsum dolor sit amet
    \item consectetur adipiscing elit.
    \item Aliquam dignissim blandit est, in dictum tortor gravida eget. In ac rutrum magna.
  \end{itemize}

\end{document}
